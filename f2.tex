\section{Bewertung von Zahlungsströmen, Anleihen}

\textbf{Zeitwert des Geldes}: Wert einer Zahlung hängt von Höhe \underline{und} Zeitpunkt ab

\textbf{Anleihen} (Schuldverschreibungen, Bonds):
\begin{itemize}
	\item typische Formen der Fremdkapitalfinanzierung
	\item verbriefte, handelbare finanzielle Ansprüche (Zahlung des Nennbetrags und Zinsen) gegenüber einem Schuldner ($=$ \textbf{Emittent})
	\item zur langfristigen Finanzierung von Unternehmen
\end{itemize}
\bigskip
\textbf{Arten von Anleihen}:
\begin{itemize}
	\item \textbf{Kuponanleihe}: Periodische fixe Zinszahlungen ($=$ Kupons) bis zur endfälligen Tilgungszahlung
	\item \textbf{Nullkuponanleihen} (Zerobond): Keine periodischen Zinszahlungen, Tilgungsbetrag wird am Laufzeitende ausgezahlt
	\item \textbf{Floating Rate Notes}: Periodische variable Zinszahlungen, die sich an den jeweils vorherrschenden kurzfristigen Zinsen orientieren, mit endfälliger Tilgungszahlung
	\item Hybride Formen mit Eigen- und Fremdkapitalcharakter: z.B. Wandelanleihen oder Optionsanleihen
\end{itemize}
Unterjährig aufgelaufene Zinsen (\textbf{Stückzinsen}) sind beim Kauf an den Verkäufer zu entrichten (damit der Käufer nicht zu viele Zinsen am Ende des Jahres erhält)\\

\textbf{Rating von Anleihen}: Bonitätsbeurteilung eines Emittenten
\begin{itemize}
	\item externes Rating durch unabhängige Agenturen
	\item Informieren aller Marktteilnehmer
	\item Ermöglichung des Erwerbs der Anleihe durch regulierte Institutionen
	\item Ratings werden in Form von Ratingklassen organisiert (AAA bis CCC-), Risikoaufschlag auf die
	Zinszahlungen bei hohem Risiko
\end{itemize}
\bigskip
\textbf{Weitere Gestaltungsmöglichkeiten von Anleihen}:
\begin{itemize}
	\item Sicherheiten: Anleihen können besichert sein (z.B. durch Vermögensgegenstände wie Aktien)
	\item Covenants: Zusatzvereinbarungen, die dem Emittenten bestimmte Handlungen erzwingen oder verbieten
	\item Kündigungsrecht: Emittent/Gläubiger haben das Recht, vorzeitig die Anleihe zu kündigen
\end{itemize}
\bigskip
\textbf{Bewertung von Anleihen}: Wert einer Anleihe zu einem bestimmten Zeitpunkt entspricht dem Wert des Zahlungsstroms zu diesem Zeitpunkt $\rightarrow$ \textbf{Zeitwert des Geldes} berücksichtigt Abzinsung/Diskontierung des Geldes

\textbf{Einperiodige Verzinsung}:
\begin{itemize}
	\item Endwert $\text{EW}=C_0\cdot (1+r)$, $\;\;\;\;\; C_T$: Zahlung im Zeitpunkt $T$, $r$: Zinssatz
	\item Barwert $\text{BW}=\cfrac{C_1}{1+r}$
\end{itemize}

\textbf{Mehrperiodige Verzinsung} bei $T$ Perioden:
\begin{itemize}
	\item $\text{EW}=C_0\cdot (1+r)^T$
	\item $\text{BW}=\cfrac{C_T}{(1+r)^T}$
\end{itemize}

\textbf{Unterjährige Verzinsung} ($m$-maliges Verzinsen pro Periode):
\begin{itemize}
	\item $\text{EW}=C_0\cdot (1+\frac{r}{m})^{m\cdot T}$
	\item Effektiver Jahreszinssatz $\text{EJZ}=\sqrt[T]{\frac{\text{EW}}{C_0}}-1$
\end{itemize}

\textbf{Stetige Verzinsung}:
\begin{itemize}
	\item $\text{EW}=C_0\cdot e^{r\cdot T}$
\end{itemize}

\textbf{Mehrperiodige Zahlungsströme}, d.h. pro Periode werden Zahlungen $C_i$ getätigt:
\begin{itemize}
	\item $\text{EW}=\sum\limits_{i=1}^{T}C_i\cdot (1+r)^{T-i}$
	\item $\text{BW}=\sum\limits_{i=1}^{T}\cfrac{C_i}{(1+r)^i}$
\end{itemize}

\textbf{Rentenformeln} für mehrperiodige Zahlungsströme:
\begin{itemize}
	\item \textbf{Ewige Rente} (periodisch konstante Zahlungen $C$ für unendliche Anzahl an Perioden): $\text{BW}=\cfrac{C}{r}$
	\item \textbf{Endliche Rente} (periodisch konstante Zahlungen $C$ für $T$ Perioden):\newline $\text{BW}=\cfrac{C}{r}\cdot \left(1-\cfrac{1}{(1+r)^T}\right)$
	\item \textbf{Ewige Rente mit konstantem Wachstum} (Ewige Rente mit periodisch wachsenden Zahlungen um Rate $g$):\\
	$\text{BW}=\cfrac{C}{r-g}$, $\;\;\;\;\;\;$wenn $g<r$
	\item \textbf{Endliche Rente mit konstantem Wachstum} (Endliche Rente mit periodisch wachsenden Zahlungen um Rate $g$):\\
	$\text{BW}=\cfrac{C}{r-g}\cdot\left(1-\left(\cfrac{1+g}{1+r}\right)^T\right)$
\end{itemize}

\textbf{Risikolose Anleihen}:
\begin{itemize}
	\item \textbf{Zerobonds} (Nullkuponanleihe, d.h. festgelegter Nennwert am Ende ausgezahlt):\\
	\textbf{Interner Zinssatz/Yield} $y=r=\sqrt[T]{\cfrac{\text{Nennwert}}{\text{Preis}}}-1$
	\item \textbf{Kuponanleihe} (Rückzahlung des Nennwertes \& zwischenzeitliche Zinszahlungen/Kupons $K_i$):\\
	\textbf{Preis} $P=\cfrac{K_1}{1+r}+\cfrac{K_2}{(1+r)^2}+\ldots+\cfrac{K_T+\text{Nennwert}}{(1+r)^T}$ 
\end{itemize}

\textbf{Dynamischer Verlauf von Anleihepreisen}:
\begin{itemize}
	\item Wert fällt im zeitlichen Verlauf, wenn $\text{Preis}>\text{Nennwert}$
	\item Wert steigt im zeitlichen Verlauf, wenn $\text{Preis}<\text{Nennwert}$
\end{itemize}
