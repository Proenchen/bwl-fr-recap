\section{Bewertung von Zahlungsströmen, Anleihen}

\textbf{Zeitwert des Geldes}: Wert einer Zahlung hängt von Höhe \underline{und} Zeitpunkt ab

\textbf{Anleihen} (Schuldverschreibungen, Bonds):
\begin{itemize}
	\item typische Formen der Fremdkapitalfinanzierung
	\item verbriefte, handelbare finanzielle Ansprüche (Zahlung des Nennbetrags und Zinsen) gegenüber einem Schuldner ($=$ \textbf{Emittent})
	\item zur langfristigen Finanzierung von Unternehmen
\end{itemize}
\bigskip
\textbf{Arten von Anleihen}:
\begin{itemize}
	\item \textbf{Kuponanleihe}: Periodische fixe Zinszahlungen ($=$ Kupons) bis zur endfälligen Tilgungszahlung
	\item \textbf{Nullkuponanleihen} (Zerobond): Keine periodischen Zinszahlungen, Tilgungsbetrag wird am Laufzeitende ausgezahlt
	\item \textbf{Floating Rate Notes}: Periodische variable Zinszahlungen, die sich an den jeweils vorherrschenden kurzfristigen Zinsen orientieren, mit endfälliger Tilgungszahlung
	\item Hybride Formen mit Eigen- und Fremdkapitalcharakter: z.B. Wandelanleihen oder Optionsanleihen
\end{itemize}
Unterjährig aufgelaufene Zinsen (\textbf{Stückzinsen}) sind beim Kauf an den Verkäufer zu entrichten (damit der Käufer nicht zu viele Zinsen am Ende des Jahres erhält)\\

\textbf{Rating von Anleihen}: Bonitätsbeurteilung eines Emittenten
\begin{itemize}
	\item externes Rating durch unabhängige Agenturen
	\item Informieren aller Marktteilnehmer
	\item Ermöglichung des Erwerbs der Anleihe durch regulierte Institutionen
	\item Ratings werden in Form von Ratingklassen organisiert (AAA bis CCC-), Risikoaufschlag auf die
	Zinszahlungen bei hohem Risiko
\end{itemize}
\bigskip
\textbf{Weitere Gestaltungsmöglichkeiten von Anleihen}:
\begin{itemize}
	\item Sicherheiten: Anleihen können besichert sein (z.B. durch Vermögensgegenstände wie Aktien)
	\item Covenants: Zusatzvereinbarungen, die dem Emittenten bestimmte Handlungen erzwingen oder verbieten
	\item Kündigungsrecht: Emittent/Gläubiger haben das Recht, vorzeitig die Anleihe zu kündigen
\end{itemize}


