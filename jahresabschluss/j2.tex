\section{Rechtliche Grundlagen der Bilanzierung}

\textbf{Handelsrecht} = Sonderrecht der Kaufleute

\textbf{Kaufmann} = Handelsgesellschaften kraft ihrer \textbf{Rechtsform} (Einzelunternehmen, Personen-/ Kapitalgesellschaften, Genossenschaften, Vereine/Stiftungen)

$\rightarrow$ Rechtsgrundlage für kaufmännischen Rechts- und Geschäftsverkehr ist das \textbf{HGB}
Unterscheidung zwischen:
\begin{itemize}
	\item \textbf{Rechtsformunabhängige} Rechnungslegungsvorschriften (1. Abschnitt HGB)
	\item \textbf{Rechtsformabhängige} Rechnungslegungsvorschriften (2. Abschnitt HGB)
\end{itemize}
\textbf{Weitere Gesetze für Rechnungslegungsvorschriften}:
\begin{itemize}
	\item AktG für Aktiengesellschaften (z.B. Vorschriften zur Bilanz, GuV und Anhang, Organisation und Buchführung, Verwendung des Jahresüberschusses)
	\item GmbHG für GmbH (z.B. Pflicht zur ordnungsgemäßen Buchführung durch
	Geschäftsführer, Ausweispflichten im JA, Vorlage des JA und Lageberichts)
	\item Vorschriften des Publizitätsgesetzes: Aufstellung von JA und Lagebericht
\end{itemize}
\bigskip
\textbf{Grundsätze ordnungsgemäßer Buchführung (GoB)}: 
\begin{itemize}
	\item \textbf{GoB} = System von allgemein anerkannten Regeln zur Buchführung, Inventarisierung, Bilanzierung, Bewertung und Ausweis
	\item GoB keine Rechtsnorm, hat aber Rechtsnormcharakter und soll Tatbestände auffangen, die der Gesetzgeber nicht geregelt hat
	\item Unterteilung in GoD, GoI, GoBil
	\item In Deutschland: \textbf{Gläubigerschutz} ist wesentl. Zweck des JA $\rightarrow$ Vorsichtsprinzip\\
	\textbf{Informationszweck} erst im Laufe der Zeit wichtiger
\end{itemize}

\textbf{Grundsätze ordnungsgemäßer Dokumentation (GoD)}:
\begin{itemize}
	\item \textbf{GoD} = Regelungen für eine zugriffsichere Aufzeichnung buchungspflichtiger Geschäftsvorfälle
	\item \textbf{Formelle GoD}: Grundsatz der Klarheit und Übersichtlichkeit, Sicherheit, formellen Kontinuität
	\item \textbf{Materielle GoD}: Grundsatz der Vollständigkeit, Richtigkeit
\end{itemize}

\textbf{Grundsätze ordnungsgemäßer Inventur (GoI)}:
\begin{itemize}
	\item \textbf{Inventur} = Bestandsaufnahme aller Vermögensgegenstände und Schulden zum Bilanzstichtag
	\item Inventur verpflichtend und Unterscheidung von drei \textbf{Inventursystemen}:
	Stichtagsinventur, Vor- oder nachgelagerte Stichtagsinventur, Permanente Inventur
\end{itemize}

\textbf{Grundsätze ordnungsgemäßer Bilanzierung (GoBil)}:
\begin{itemize}
	\item \textbf{Formelle GoBil}: Grundsatz der$\ldots$
	\begin{itemize}
		\item Klarheit \& Übersichtlichkeit: Mindestgliederung in Aktiva = AV + UV + RAP und Passiva = S + EK + RAP
		\item formellen Bilanzkontinuität: Beibehaltung der Gliederung und des Abschlussstichtages, Einhaltung der \textbf{Bilanzidentität}, d.h. Anfangsbilanz = Schlussbilanz des Vorjahres
	\end{itemize}
	\item \textbf{Materielle GoBil}: Grundsatz der Vollständigkeit, Richtigkeit, materiellen Bilanzkontinuität
\end{itemize}

\textbf{Grundsatz der Vorsicht als zentraler GoB}: Dient dem Gläubigerschutz. Konkretisierung durch weitere Prinzipien:
\begin{itemize}
	\item \textbf{Realisationsprinzip}: Gewinne erst dann ausgewiesen, wenn sie am Abschlussstichtag realisiert sind
	\item \textbf{Imparitätsprinzip}: Alle vorhersehbaren Risiken und Verluste sind zu berücksichtigen und auszuweisen. Ausweisverbot für unrealisierte Gewinne.
	\item \textbf{Anschaffungs- und Herstellungskostenprinzip}: Vermögensgegenstände höchstens mit  Anschaffungs-/Herstellkosten und Rückstellungen mit Erfüllungsbetrag ansetzen
	\item \textbf{Niederstwertprinzip (NWP)}: Bei zwei für die Bewertung von Aktiva in Betracht kommenden Werten ist der niedrigere Wert anzusetzen.
	\item Grundsatz der Einzelbewertung: Gewährleistet Wirksamkeit des Imparitätsprinzips
\end{itemize}

\textbf{Grundsatz der Periodenabgrenzung}:
\begin{itemize}
	\item Basiert auf \textbf{Verursachungsprinzip}: Aufwendungen und Erträge des Geschäftsjahrs sind \underline{unabhängig} von den Zeitpunkten der Zahlungen im JA zu berücksichtigen.
	\item \textbf{Abgrenzung der Sache nach}: Ergänzt Realisationsprinzip. Gegenüberstellung von realisierten Erträgen und ihren Aufwendungen
	\item \textbf{Abgrenzung der Zeit nach}: wenn Erträge und Aufwendungen weder nach dem Realisationsprinzip noch der Sache nach zugeordnet werden können
	\item Streng \textbf{zeitraumbezogene Beträge} (z.B. Mieten) sind anteilig den betreffenden Geschäftsjahren zuzuordnen
\end{itemize}