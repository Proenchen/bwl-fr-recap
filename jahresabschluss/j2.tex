\section{Rechtliche Grundlagen der Bilanzierung}

\textbf{Handelsrecht} = Sonderrecht der Kaufleute

\textbf{Kaufmann} = Handelsgesellschaften kraft ihrer \textbf{Rechtsform} (Einzelunternehmen, Personen-/ Kapitalgesellschaften, Genossenschaften, Vereine/Stiftungen)

$\rightarrow$ Rechtsgrundlage für kaufmännischen Rechts- und Geschäftsverkehr ist das \textbf{HGB}
Unterscheidung zwischen:
\begin{itemize}
	\item \textbf{Rechtsformunabhängige} Rechnungslegungsvorschriften (1. Abschnitt HGB)
	\item \textbf{Rechtsformabhängige} Rechnungslegungsvorschriften (2. Abschnitt HGB)
\end{itemize}
\textbf{Weitere Gesetze für Rechnungslegungsvorschriften}:
\begin{itemize}
	\item AktG für Aktiengesellschaften (z.B. Vorschriften zur Bilanz, GuV und Anhang, Organisation und Buchführung, Verwendung des Jahresüberschusses)
	\item GmbHG für GmbH (z.B. Pflicht zur ordnungsgemäßen Buchführung durch
	Geschäftsführer, Ausweispflichten im JA, Vorlage des JA und Lageberichts)
	\item Vorschriften des Publizitätsgesetzes: Aufstellung von JA und Lagebericht
\end{itemize}
\bigskip
\textbf{Grundsätze ordnungsgemäßer Buchführung (GoB)}: 
\begin{itemize}
	\item \textbf{GoB} = System von allgemein anerkannten Regeln zur Buchführung, Inventarisierung, Bilanzierung, Bewertung und Ausweis
	\item GoB keine Rechtsnorm, hat aber Rechtsnormcharakter und soll Tatbestände auffangen, die der Gesetzgeber nicht geregelt hat
	\item Unterteilung in GoD, GoI, GoBil
	\item In Deutschland: \textbf{Gläubigerschutz} ist wesentl. Zweck des JA $\rightarrow$ Vorsichtsprinzip\\
	\textbf{Informationszweck} erst im Laufe der Zeit wichtiger
\end{itemize}

\textbf{Grundsätze ordnungsgemäßer Dokumentation (GoD)}: