\section{Bilanzierungsgrundsätze und –vorschriften}

\textbf{Bilanzierungsgrundsätze}: \textbf{Frage}: Welche Güter in die Bilanz aufnehmen?\\
$\rightarrow$ \textbf{Bilanzansatz}: Aufteilung nach Pflicht, Wahlrecht und Verbot

\textbf{Bilanzierungsfähigkeit}: Eigenschaft eines Sachverhaltes in der Bilanz angesetzt zu werden
\begin{itemize}
	\item \textbf{abstrakte Bilanzierungsfähigkeit}: Prüfung auf grundsätzliche Eignung als Bilanzansatz losgelöst vom konkreten Einzelfall. Gegeben, wenn Objekt aus einer der Kategorien:
	\begin{itemize}
		\item Vermögensgegenstand (hat wirtschaftlichen Wert, ist bewertbar und verkehrsfähig)
		\item Schulden, Verbindlichkeiten, Rückstellungen (sicher erwartete Belastungen des Vermögens, erzeugen Leistungsverpflichtung und sind bewertbar)
		\item Eigenkapital
		\item Rechnungsabgrenzungsposten
	\end{itemize}
	$\rightarrow$ Definition von Vermögen und Schulden existiert nicht $\rightarrow$ Ableitung aus GoB
	\item \textbf{Konkrete Bilanzierungsfähigkeit}: Wenn abstrakte Bilanzierungsfähigkeit gegeben, muss
	konkrete Bilanzierungsfähigkeit geprüft werden. Ist das Objekt im konkreten Einzelfall bilanzierungsfähig? $\rightarrow$ Verlangt Erfüllung von drei Kriterien:
	\begin{itemize}
		\item \textbf{Persönliche Zuordnung}: Objekt muss dem Bilanzierenden wirtschaftlich zuzurechnen sein
		\item \textbf{Sachliche Zuordnung}: Objekt muss dem Unternehmensbetrieb zuzurechnen sein 
		\item \textbf{Kein Bilanzierungsverbot}
	\end{itemize}
	Randfälle, in denen rechtliches und wirtschaftliches Eigentum auseinanderfallen: 
	\begin{itemize}
		\item Ware unter Eigentumsvorbehalt: Bilanzierung beim Käufer
		\item Vermögensgegenstände, zur Sicherung abgetretene Forderungen: Bilanzierung beim Sicherungsgeber
		\item Einkaufskommissionsware: Bilanzierung beim Kommittenten
	\end{itemize}
\end{itemize}
\bigskip
\textbf{Rechtsformunabhängige Bilanzierungsvorschriften}:
\begin{itemize}
	\item Wenn Bilanzierungsfähigkeit gegeben, dann gilt Bilanzierungspflicht nach Grundsatz der Vollständigkeit (Ausnahme: Bilanzierungswahlrecht, Bilanzierungsverbot)
	\item Bei Unternehmenskauf: Ausweispflicht für \textbf{Goodwill} = Kaufpreis$-$Substanzwert (Wert des EK) als immatrieller Vermögensgegenstand
	\item Ansatzpflicht für transistorische RAP (antizipative RAP unter sonstige Forderungen/Verbindlichkeiten)
	\item \textbf{Aktivierungswahlrecht} für selbst geschaffene immaterielle Vermögensgegenstände (nach FS3/17) und \textbf{Disagio} darf unter RAP aktiviert werden
	\item Aktivierungsverbot für Forschungskosten und Aufwendungen für Gründung eines Unternehmens, Beschaffung von Eigenkapital, Abschluss von Versicherungsverträgen
	\item Aktivierungswahlrecht für Entwicklungskosten
\end{itemize}
\bigskip
\textbf{Ausweis- und Gliederungsvorschriften}:
\begin{itemize}
	\item Keine Gliederungsvorschriften für Nicht-Kapitalgesellschaften, nur Ausweis nötig
	\item Bei Ausweis und Gliederung sind die GoB zu beachten
	\item Bilanz kann in Konto- oder Staffelform aufgestellt sein
	\item Für Kapitalgesellschaften und bestimmte Personenhandelsgesellschaften: Verbindliches Gliederungsschema (abhängig von Größe des Unternehmens) in Kontoform 
\end{itemize}

\textbf{Grundsätze für die Gliederung der Bilanz einer Kapitalgesellschaft}:
\begin{itemize}
	\item Stetigkeit der Darstellung
	\item Ausweis der Vorjahresbeträge zu den jeweiligen Posten
	\item Weitere Untergliederung ist unter Beachtung der vorgeschriebenen Gliederung zulässig
	\item Hinzufügung neuer Posten darf nur vorgenommen werden, wenn ihr Inhalt nicht von einem vorgeschriebenen Posten gedeckt wird
	\item Abweichende Gliederung nur, wenn die für die Übersichtlichkeit erforderlich ist
	\item Ausweis von Posten ohne Betrag nur, wenn Vorjahresbilanz einen Betrag für diesen Posten ausweist
\end{itemize}

\textbf{Kapitalmarktorientierte Kapitalgesellschaften}: Wenn deren Wertpapiere an einem
organisierten Markt zum Handel zugelassen oder beantragt wurde.\\
\textbf{Beispiel-Gliederung}: FS3/29-33 (IM ERNST AUSWENDIG LERNEN????)\\

\textbf{Bilanzierungsvorschriften für bestimmte Posten}
\begin{itemize}
	\item Rechtsformunabhängig: Rückstellungen, Haftungsverhältnisse
	\item Rechtsformabhängig: Beteiligungen, verbundene Unternehmen, Eigenkapital, Latente Steuern
\end{itemize}

\textbf{Bilanzierung von Rückstellungen}:
\begin{itemize}
	\item \textbf{Statische Interpretation}: Rückstellungen aufgrund einer Verpflichtung (z.B. wirtschaftlich oder rechtlich)
	\item \textbf{Dynamische Interpretation}: Rückstellungen für Aufwendungen
	\item Passivierungspflichtige Rückstellungen im HGB definiert
	\item Passivierungsverbot von nicht im HGB genannten Rückstellungen
	\item KapG und bestimmte PersG müssen folgende Posten ausweisen: Rückstellungen für Pensionen, Steuerrückstellungen, Sonstige Rückstellungen. Kleine KapG brauchen nur einen Posten
	\item Vermerk von Haftungsverhältnissen
\end{itemize}

\textbf{Bilanzierung von Beteiligungen an verbundenen Unternehmen}: Ausweis unter Finanzanlagen. Aufgliederung der Finanzanlagen nach
\begin{enumerate}
	\item Art der Investition - Anlagen oder Ausleihungen
	\item Möglichkeiten der Einflussnahme auf die Unternehmen
\end{enumerate}
\bigskip
\textbf{Bilanzierung des Eigenkapitals}: Ausweis gem. Gliederungsschema der Bilanz, Zusatzangaben bei AGs

\textbf{Bilanzierung latenter Steuern}: \textbf{Steuerliches Abgrenzungsproblem} stellt sich, wenn Handels- und Steuerbilanzgewinn voneinander abweichen. HGB sieht Konzept der latenten Steuerabgrenzung vor:
\begin{itemize}
	\item \textbf{Passivierungspflicht}: Zukünftige Steuerbelastung ist als passive latente Steuer abzugrenzen
	\item \textbf{Aktivierungswahlrecht}: Bei zukünftiger Steuerentlastung kann eine aktive latente Steuer ausgewiesen werden
\end{itemize}