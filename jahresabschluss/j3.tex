\section{Bilanzierungsgrundsätze und –vorschriften}

\textbf{Bilanzierungsgrundsätze}: \textbf{Frage}: Welche Güter in die Bilanz aufnehmen?\\
$\rightarrow$ \textbf{Bilanzansatz}: Aufteilung nach Pflicht, Wahlrecht und Verbot

\textbf{Bilanzierungsfähigkeit}: Eigenschaft eines Sachverhaltes in der Bilanz angesetzt zu werden
\begin{itemize}
	\item \textbf{abstrakte Bilanzierungsfähigkeit}: Prüfung auf grundsätzliche Eignung als Bilanzansatz losgelöst vom konkreten Einzelfall. Gegeben, wenn Objekt aus einer der Kategorien:
	\begin{itemize}
		\item Vermögensgegenstand (hat wirtschaftlichen Wert, ist bewertbar und verkehrsfähig)
		\item Schulden, Verbindlichkeiten, Rückstellungen (sicher erwartete Belastungen des Vermögens, erzeugen Leistungsverpflichtung und sind bewertbar)
		\item Eigenkapital
		\item Rechnungsabgrenzungsposten
	\end{itemize}
	$\rightarrow$ Definition von Vermögen und Schulden existiert nicht $\rightarrow$ Ableitung aus GoB
	\item \textbf{Konkrete Bilanzierungsfähigkeit}: Wenn abstrakte Bilanzierungsfähigkeit gegeben, muss
	konkrete Bilanzierungsfähigkeit geprüft werden. Ist das Objekt im konkreten Einzelfall bilanzierungsfähig? $\rightarrow$ Verlangt Erfüllung von drei Kriterien:
	\begin{itemize}
		\item \textbf{Persönliche Zuordnung}: Objekt muss dem Bilanzierenden wirtschaftlich zuzurechnen sein
		\item \textbf{Sachliche Zuordnung}: Objekt muss dem Unternehmensbetrieb zuzurechnen sein 
		\item \textbf{Kein Bilanzierungsverbot}
	\end{itemize}
	Randfälle, in denen rechtliches und wirtschaftliches Eigentum auseinanderfallen: 
	\begin{itemize}
		\item Ware unter Eigentumsvorbehalt: Bilanzierung beim Käufer
		\item Vermögensgegenstände, zur Sicherung abgetretene Forderungen: Bilanzierung beim Sicherungsgeber
		\item Einkaufskommissionsware: Bilanzierung beim Kommittenten
	\end{itemize}
\end{itemize}



