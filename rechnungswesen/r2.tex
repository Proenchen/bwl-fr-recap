\section{Bilanz und GuV im Accounting Equation Approach}

\textbf{Bilanz}: Momentaufnahme der wirtschaftlichen Situation eines Unternehmens zu einem bestimmten
Zeitpunkt
\begin{center}
	\includegraphics[width=0.7\textwidth]{images/bilanz.png}
\end{center}

\textbf{Vermögensgegenstände}: Repräsentieren zukünftige ökonomische Werte, die dem Unternehmen zufließen und zuverlässig in Geldeinheiten messbar sind
\begin{itemize}
	\item \textbf{Anlagevermögen} (langfristig): Sachanlagen, Finanzanlagen, Immaterielle Vermögensgegenstände, Aktive latente Steuern
	\item \textbf{Umlaufvermögen} (kurzfristig): Vorräte, Geleistete Anzahlungen, Forderungen, Wertpapiere, Liquide Mittel
	\item Vermögensgegenstände oft nur zu historischen Anschaffungskosten bewertet oder noch niedriger (Abschreibungen)
	\item Wertvolle Unternehmenseigenschaften, wie z.B. Humankapital (Fähigkeiten und Erfahrung der Mitarbeiter), Verträge über Großaufträge, Reputation und Marke bleiben unberücksichtigt!
	$\rightarrow$ Vorsichtsprinzip
\end{itemize}
\pagebreak
\textbf{Fremdkapital}: \textbf{Schulden}, d.h. Verpflichtungen des Unternehmens um Geld, Güter und Dienstleistungen an einen externen Anspruchsberechtigten bereitzustellen, die zuverlässig in Geldeinheiten messbar sind.
Beispiele:
\begin{itemize}
	\item Verbindlichkeiten, z.B. Kredit, Verpflichtungen, Steuern, Höhe und Fälligkeit stehen fest
	\item Rückstellungen, z.B. Pensionen, Höhe und Fälligkeit werden geschätzt
	\item Muss auf dem Matching Principle genügen!
\end{itemize}
\bigskip
\textbf{Eigenkapital}: Reinvermögen des Unternehmens (= Wert der Vermögensgegenstände $-$ Wert der Schulden) und entspricht dem Unternehmenswert aus Perspektive der Eigentümer.
Besteht aus:
\begin{itemize}
	\item Gezeichnetes Kapital $+$ Kapitalrücklage $\rightarrow$ eingezahlt von den Gesellschaftern
	\item Gewinnrücklage $\rightarrow$ kumuliert aus bisherigen Gewinnen
\end{itemize}
\bigskip

\begin{wrapfigure}{r}{4.5cm}
	\includegraphics[width=0.3\textwidth]{images/guv-ex.png}
	\vspace{-100pt}
\end{wrapfigure} 

\textbf{Gewinn- und Verlustrechnung (GuV)}:
\begin{itemize}
	\item Wirtschaftlicher Erfolg = Umsatzerlöse (Realisationsprinzip) $-$ Aufwendungen (Matching Principle)
	\item Zwei Arten von Aufwendungen:
	\begin{itemize}
		\item Herstellungskosten (sachliche Abgrenzung)
		\item Periodenaufwendungen (zeitliche Abgrenzung)
	\end{itemize}
	\item Nur Veränderungen des EK aus gewöhnlichen\\ Geschäftstätigkeiten gehen in die GuV ein!\\
	Interaktionen mit Eigentümern, wie z.B. Dividendenauszahlung irrelevant.
\end{itemize}
\bigskip
\textbf{Grundformen der Verbuchung}:
\begin{itemize}
	\item \textbf{Antizipative Rechnungsabgrenzung}: Vorwegnahme von Aufwendungen wofür Zahlungen erst in einer späteren Periode erfolgen (3 und 5).
	\item \textbf{Transitorische Rechnungsabgrenzung}: Aufwendungen und Umsatzerlöse in nächste Periode übertragen, obwohl Zahlungen bereits erfolgt (2 und 6).
\end{itemize}
\begin{center}
	\includegraphics[width=0.8\textwidth]{images/gf1.png}
\end{center}
\begin{center}
	\includegraphics[width=0.8\textwidth]{images/gf2.png}
\end{center}

\textbf{Worksheet Approach} Bilanz und GuV: s. FS2/21-28, \underline{ÜBEN}!
\begin{itemize}
	\item Erste Zeile Bilanz: \textbf{Eröffnungsbilanz}, Erste Zeile GuV: leer
	\item $\text{Gewinn}=\text{EK}_\text{Schlussbilanz}–\text{EK}_\text{Eröffnungsbilanz}$
	\item Für jeden Geschäftsvorfall gilt:
	\begin{itemize}
		\item $\Delta L + \Delta V=\Delta E+\Delta F$ \\
		$L$: Liquide Mittel, $V$: Nicht-liquide Vermögensgegenstände, $E$: EK, $F$: FK
		\item $\Delta E=U-A$\\
		$U$: Umsatzerlöse, $A$: Aufwendungen
		\item Passiva und Aktiva gleich verändert
	\end{itemize}
\end{itemize}
\begin{center}
	\includegraphics[width=0.8\textwidth]{images/sze.png}
\end{center}
\textbf{Häufige Geschäftsvorfälle}:
\begin{center}
	\includegraphics[width=0.5\textwidth]{images/gsv.png}
\end{center}
\bigskip

\textbf{T-Konten}: \textit{(s.FS2/49)} Probleme:
\begin{itemize}
	\item Änderungen aus einem Geschäftsvorfall, werden nicht im Zusammenhang gezeigt
	\item Konten aus der Bilanz und der GuV werden vermischt
	\item Anstatt $+$ und $–$, wird \enquote{Soll} und \enquote{Haben} benutzt, aber Wertänderungen sind nicht \enquote{logisch} nachvollziehbar
\end{itemize}
\bigskip
\textbf{Kapitalflussrechnung}: Erfasst Ursachen für Veränderungen der Liquiden Mittel in einer Periode
\begin{center}
	\includegraphics[width=0.9\textwidth]{images/kfr.png}
\end{center}
Manchmal werden Zahlungsströme am Ende aus Bilanz und GuV abgeleitet $\rightarrow$ \textbf{derivative Ermittlung}

\pagebreak
\textbf{Netto-Geldvermögen}: Liquide Mittel + Wertpapiere + Forderungen $–$ kurzfristige Schulden
\begin{itemize}
	\item \textbf{Einnahme}: Netto-Geldvermögen steigt
	\item \textbf{Ausgabe}: Netto-Geldvermögen sinkt
\end{itemize}
\bigskip
\textbf{Net Working Capital} (Netto-Umlaufvermögen): Netto Geldvermögen + Vorräte = Umlaufvermögen $–$ kurzfristige Schulden\\

\textbf{Rechnungsabgrenzungsposten} (RAP):
\begin{itemize}
	\item Geleistete Anzahlungen sind oft am Ende der Periode \enquote{verbraucht}. RAP-Buchungsvariante:
	\begin{enumerate}
		\item Zuerst alle Aufwendungen in der Periode buchen
		\item Anteiliges Zurückdrehen der Kosten, die eigentlich erst in der nächsten Periode geleistet werden
		$\rightarrow$ Aufwendungen teilweise in neue Periode übertragen
	\end{enumerate}
	$\rightarrow$ \textbf{aktiver Rechnungsabgrenzungsposten}
	\item Verpflichtungen, die durch erhaltene Anzahlungen entstehen, sind oft am Ende der Periode \enquote{erledigt}. RAP-Buchungsvariante:
	\begin{enumerate}
		\item Zuerst alle Erträge in der Periode buchen
		\item Anteiliges Zurückdrehen der Erträge, die eigentlich erst in der nächsten Periode geleistet werden
		$\rightarrow$ Erträge teilweise in neue Periode übertragen
	\end{enumerate}
	$\rightarrow$ \textbf{passiver Rechnungsabgrenzungsposten}
	\item \textbf{Pro}: \enquote{Zurückdrehen} kann oft vermieden werden, \textbf{Contra}: Bilanz und GuV zwischendurch nicht genau
	\item Betrifft nur Vermögensgegenstände und Schulden, die erst am Periodenende gebildet werden, um zeitraumbezogene Aufwendungen und Erträge zu \enquote{korrigieren}!
\end{itemize}