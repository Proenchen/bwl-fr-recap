\section{Kostenrechnung}
Übersicht: s. FS4/13
\subsection{Kostenartenrechnung}
\textbf{Frage}: Welche Kosten sind in welcher Höhe in einer Periode angefallen? 
$\rightarrow$ Aufteilung auf Kostenarten

\textbf{Kostenarten}: Kategorie von Kosten, die nach bestimmten Kriterien aufgegliedert werden können

\textbf{Kostenartenhauptgruppen}: Personal- und Sozialkosten, Sach- und Materialkosten
Dienstleistungskosten, Kosten für Lizenzen, Kapitalkosten, öffentliche Abgaben und Steuern, Versicherungskosten und kalkulatorische Wagniskosten

\textbf{Variable Kosten}: Kosten, die mit der Ausprägung (Stückzahl) eines Kostentreibers variieren

\textbf{Fixe Kosten}: Kosten, die unabhängig vom Kostentreiber immer in konstanter Höhe anfallen

\textbf{Kostentreiber}: Variable, die am besten erklärt, wie die gesamten Kosten eines Kostenobjektes zustande kommen, z.B. Produktionsvolumen, Lieferungen, $\ldots$

\textbf{Break-even-Analyse}:
\begin{center}
	\includegraphics[width=\textwidth]{images/be-analyse.png}
\end{center}
\textit{Bsp. s. FS4/21}

