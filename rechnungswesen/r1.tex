\section{Grundlagen des externen Rechnungswesens}

\textbf{Abgrenzung zur Finanzwirtschaft}:
\begin{itemize}
	\item Finanzwirtschaft: Fokus auf dem Wert von Investitionen über deren \underline{gesamte Lebensdauer}
	\item Rechnungswesen: Fokus auf wirtschaftlichem Erfolg eines Unternehmens während einer \underline{abgegrenzten Periode}
	\item Daten des Rechnungswesens im Gegensatz zu Zahlungsströmen nicht eindeutig!
\end{itemize}
\bigskip
\textbf{Reinvermögen und wirtschaftlicher Erfolg}:
\begin{itemize}
	\item \textbf{Reinvermögen} zum Zeitpunkt $t_i$ 
	\\= Wert der Vermögensgegenstände bei $t_i - \text{ Wert der Schulden bei } t_i$ 
	\item \textbf{Wirtschaftlicher Erfolg} der Periode $t_i$: Änderung des Reinvermögens zwischen Anfang ($t_{i-1}$) und Ende ($t_i$) der Periode 
\end{itemize}
\bigskip
\textbf{Erfolgsmessung auf Basis von Marktwerten}:
Wie können wir den Erfolg eines Unternehmens messen?
Ansätze:
\begin{itemize}
	\item Veränderung der Marktwerte der Vermögensgegenstände und Schulden\\
	Probleme bei Marktwerten von Vermögensgegenständen:
	\begin{itemize}
		\item Für viele VG irrelevant, da Unternehmen nicht vor hat, diese zu verkaufen
		\item Eindeutige Bestimmung oft nicht möglich
		\item Gefahr von subjektiven Schätzungen
		\item Unklar, wann Gewinn oder Verlust gebucht werden kann
	\end{itemize}
	Probleme bei Marktwerten von Schulden:
	\begin{itemize}
		\item Nur beobachtbar, wenn Schulden in Form von handelbaren Wertpapieren vorliegen
		\item Sinnlos, Schulden zum Marktwert zurückzukaufen (Zinseffekte)
	\end{itemize}
	\item Veränderung des Börsenwertes\\
	Probleme:
	\begin{itemize}
		\item Nur möglich, wenn Eigenkapital als Aktien an Börsen gehandelt wird
		\item Henne-Ei-Problem: Was ist zuerst: Erfolg oder neuer Börsenwert?	
	\end{itemize}
	$\rightarrow$ \textbf{Erfolgsmessung auf der Basis von Prinzipien des Rechnungswesens}
\end{itemize}
\pagebreak
\textbf{Grundideen des Rechnungswesens}
\begin{itemize}
	\item Informationsbereitstellung:
	\begin{itemize}
		\item Dokumentationsfunktion $\rightarrow$ Beurteilung der finanziellen Lage des Unternehmens
		\item Planungsfunktion $\rightarrow$ Testen von Entscheidungen auf Wirtschaftlichkeit (Soll-Werte)
		\item Kontrollfunktion $\rightarrow$ Überprüfung der Wirkung von Entscheidungen (Ist-Werte)
	\end{itemize}
	\item Rechnungswesen um Tatbestände und Vorgänge zu dokumentieren und zu messen. Unterscheide:
	\begin{itemize}
		\item \textbf{Externes Rechnungswesen}: gemessene Werte für externe Adressaten, es gelten bestimmte Regeln für die Gestaltung der Daten
		\item  \textbf{Internes Rechnungswesen}: gemessene Werte für unternehmensinterne Adressaten, Abweichung von Regeln möglich
	\end{itemize}
	\item $\textbf{Gewinn}=\text{Erträge}-\text{Aufwendungen}$ (= Verlust, wenn Gewinn $<0$)
	\item \textbf{Erträge/Umsatzerlöse} = Wert aller erbrachten Güter- und Dienstleistungen einer Periode
	\item \textbf{Aufwendungen} = Wert aller verbrauchten Güter- und Dienstleistungen einer
	Periode, die eingesetzt werden mussten, um Umsatzerlöse zu generieren
	\item \textbf{Periodisierung}: Aufwendungen und Erträge, die zusammengehören, sollen in der gleichen Periode zusammengefügt werden
\end{itemize}
\bigskip
\textbf{Rechnungslegungsprinzipien}:
\begin{itemize}
	\item \textbf{Realisationsprinzip}: Wann sollen Umsatzerlöse als realisiert gelten?\\
	$\rightarrow$ Sobald eine geschäftliche Transaktion abgeschlossen ist. Das ist der Fall, wenn:
	\begin{itemize}
		\item Messung der Umsatzerlöse und den verbundenen Aufwendungen möglich
		\item Hohe Sicherheit, dass das Unternehmen von der Transaktion profitiert (zuverlässiger Kunde)
		\item Unternehmen hat Risiken und Nutzungsvorteile auf den Transaktionspartner transferiert
	\end{itemize}
	Es ist \underline{nicht entscheidend}, ob eine Zahlung erfolgt ist!
	\item \textbf{Matching Principle}: Wann sollen Aufwendungen in die Gewinnberechnung einfließen? 
	$\rightarrow$ Sobald Umsatzerlöse realisiert werden, werden die dafür notwendigen Aufwendungen verbunden
	\begin{itemize}
		\item Umsatzerlöse und Aufwendungen fließen so in dieselbe Periode ein
		\item Nicht entscheidend, dass in dieser Periode auch eine aufwandsverbundene Auszahlung erfolgt
		\item \textbf{Sachliche Abgrenzung} (z.B. Produktverkauf)
		\item \textbf{Zeitliche Abgrenzung} (z.B. Dienstleistung/Miete)
	\end{itemize}
	\item \textbf{Vorsichtsprinzip} (Imparitätsprinzip): Wann wird eine Unsicherheit ausgewiesen?
	$\rightarrow$ Potenzielle Verluste so früh wie möglich ausweisen und Gewinne erst, wenn sie erzielt worden sind
	\item \textbf{Fortführungsprinzip}: Annahme der Fortführung des Unternehmens
	\item \textbf{Konsistenzprinzip}: Rechnungslegung über mehrere Perioden auf dieselbe Weise durchgeführt
\end{itemize}