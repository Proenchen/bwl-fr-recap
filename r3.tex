\section{Mehrdeutigkeit im externen Rechnungswesen}

\textbf{Abschreibungsmethoden}: 
\begin{itemize}
	\item relevant für Anlagevermögen
	\item Warum? Matching Principle! Verteilung des Gesamtaufwands über die Perioden, in denen der Vermögensgegenstand zu Umsatzerlösen geführt hat
	\item \textbf{Gesamtaufwand} = Einkaufspreis $–$ Restwert am Ende der letzten Periode
	\item \textbf{Buchwert} zum Zeitpunkt $t$ = Anfangswert $–$ Abschreibungen bis Zeitpunkt $t$
	\item Am Ende: Buchwert = Restwert, In der Zwischenzeit: Buchwert $\neq$ Marktwert
\end{itemize}
\bigskip
\textbf{Planmäßige Abschreibungen}: Erwartete Abschreibungen
\begin{itemize}
	\item \textbf{Lineare Abschreibung}: Höhe der Abschreibung $ABS_t$ in jeder Periode gleich.
	\begin{tightcenter}
		$ABS_t=\cfrac{AW_{VG}-LW_{VG}}{T}$
	\end{tightcenter}
	mit $AW_{VG}$ Anfangswert eines Vermögensgegenstandes $VG$ mit Nutzungsdauer von $T$ Perioden und einem Liquidationswert $LW_{VG}$
	\item \textbf{Geometrisch-degressive Abschreibung}: Höhe der Abschreibung pro Periode ist in allen
	Perioden ein konstanter Prozentsatz $p$ vom Buchwert.
	\begin{tightcenter}
		$p=1-\sqrt[T]{\cfrac{LW_{VG}}{AW_{VG}}}\qquad$ und $\qquad ABS_t=\text{Buchwert}_{t-1}\cdot p$
	\end{tightcenter}
	\item \textbf{Arithmetisch-degressive Abschreibung}: Abschreibungsbetrag pro Periode nimmt um denselben Betrag ab
	\item \textbf{Progressive Abschreibung}: Pro Periode steigende Abschreibungsbeträge
	\item \textbf{Leistungsabschreibung}: Basierend auf Leistungsabgabe in der Periode
\end{itemize}
\bigskip
\textbf{Außerplanmäßige Abschreibungen}: Unerwartete Abschreibungen, z.B. wegen Umweltkatastrophen, werden nötig, wenn planmäßige Abschreibungen nicht mehr angemessen sind, weil Buchwert weit über dem
Marktwert liegt. $\rightarrow$ Buchwert wird bis zum heutigen Marktwert gesenkt
\begin{itemize}
	\item \textbf{Zuschreibung}: Rückgängigmachen einer außerplanmäßigen Abschreibung
\end{itemize}
\bigskip
\textbf{Aktivierung}: Auszahlungen werden gleichzeitig als Vermögensgegenstände in die Bilanz aufgenommen und wirken sich nicht negativ aufs EK aus. 
Diese werden dann abgeschrieben, s.FS3/F16-17. $\rightarrow$ Matching Principle\\

\textbf{Rückstellung}: Passivierte Verpflichtungen (Schulden), für zukünftige Auszahlungen, die jetzt verursacht wurden. $\rightarrow$ sofortiger Aufwand (Matching Principle)
\begin{itemize}
	\item Höhe und Termin ungewiss, aber schätzbar
	\item Bsp.: Pensions- oder Steuerrückstellung
	\item Nicht erlaubt, um Steuern zu sparen oder man weniger Umsatz erwartet!\\
	$\rightarrow$ \textbf{Income Smoothing}
\end{itemize}
\bigskip
\textbf{Vorratsbewertung}: Bewertung der Vorräte beeinflusst die Herstellungskosten der verkauften Güter und damit den Gewinn in einer bestimmten Periode.
\begin{itemize}
	\item \textbf{Gleitend gewogenes Durchschnittsverfahren}: Betrachtung historischer Anschaffungswerte $\rightarrow$ Durchschnitt von Anfangsvorrat und Einkäufen
	\item \textbf{FIFO}: Betrachtung historischer Anschaffungskosten $\rightarrow$ Abgang \enquote{first in, first out}
	\item \textbf{LIFO}: Betrachtung historischer Anschaffungskosten $\rightarrow$ Abgang \enquote{last in, first out}
	\item \textbf{HIFO/LOFO}: \enquote{highest/lowest in, first out}
\end{itemize}
\textit{Bsp. s.FS3/F23-25}\\

\textbf{Herstellungskosten}: fallen erst an, wenn diese Produkte verkauft werden (Matching Principle)
\begin{itemize}
	\item \textbf{Teilkostenrechnung}: Nur variable Material- und Fertigungskosten gehen in die Berechnung der Herstellungskosten ein. (\textbf{variable costing} (VC))
	Fixkosten und Nicht-Fertigungskosten werden als Periodenaufwand verbucht.
	\item \textbf{Vollkostenrechnung}: Variable und fixe Fertigungskosten gehen in die Berechnung der Herstellungskosten ein.
	Nicht-Fertigungskosten werden wieder als Periodenaufwand berücksichtigt. (\textbf{absorption costing} (AC))
	$\rightarrow$ Matching Principle
\end{itemize}
\textit{Bsp. s.FS3/F29-30}