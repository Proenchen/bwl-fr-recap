\section{Methoden der Investitionsentscheidung}

\textbf{Capital Budgeting}: Entscheidungsprozess über die Durchführung einer Investition
\begin{itemize}
	\item Identification: Möglichkeiten zur Investition finden
	\item Evaluation: Projektbewertung
	\item Selection: Festlegung von Akzeptanzkriterien
	\item Implementation and follow-up: Entscheidungen über die nächsten Schritte
\end{itemize}
\bigskip
\textbf{Kapitalwertmethode}:
\begin{itemize}
	\item \textbf{Kapitalwert} $KW$ (Net Present Value) $=$ Summe aller Barwerte von zukünftigen Zahlungen abzüglich der Anfangsauszahlung
	\item Zinssatz entspricht pro Periode dem Zinssatz einer Alternative
	\item $KW = -\text{Anfangsauszahlung}+BW_\text{Zahlungen}$
	\item Führe Investitionsprojekt durch, wenn $KW > 0$
	\item \textbf{Pro}: Zahlungen als Grundlage, Mehrperiodizität, Berücksichtigung erwarteter risikoadäquater Renditen
	\item \textbf{Contra}: Ist das Kapital knapp, so können nicht alle Projekte mit positivem $KW$ durchgeführt werden und es ergibt sich keine Lösung des Auswahlproblems\\
	$\rightarrow$ \textbf{Lösung}: Ermittlung des max. Gesamtkapitalwerts innerhalb der Budgetgrenze
\end{itemize}
\bigskip
\textbf{Amortisationsrechnung}:
\begin{itemize}
	\item Bestimme Zeitspanne, in der die Anfangsauszahlung wieder durch Zahlungen zurückgeflossen ist
	\item Führe Investitionsprojekt durch, wenn eine maximale Amortisationsdauer nicht überschreitet wird
	\item \textbf{Pro}: Schnell und einfach, Eignung für Unternehmen, die keinen guten Zugang zu
	Kapitalmärkten haben, Ermöglicht Managementbewertung
	\item \textbf{Contra}: 
	\begin{itemize}
		\item Keine Berücksichtigung des Zeitpunkts der Zahlungen (s. FS3/8 Projekt 1 und 2) $\rightarrow$ \textbf{Lösung}: Betrachte stattdessen Barwerte aller Zahlungen
		\item Keine Berücksichtigung von Zahlungen nach der Amortisationsdauer (s. FS3/8 Projekt 2 und 3)
		\item Willkürliche Festlegung der Amortisationsdauer
	\end{itemize}
\end{itemize}
\bigskip
\textbf{Interne Zinssatzmethode}
\begin{itemize}
	\item \textbf{Interner Zinssatz} $=$ Zinssatz, bei dem der Kapitalwert $0$ beträgt
	\item Führe Investitionsprojekt durch, wenn der interne Zinssatz größer als ein geforderter Zinssatz ist
	\item Ermittle IZS durch Auflösung der Kapitalwertgleichung (s. FS3/13)
	\item \textbf{Pro}: Berücksichtigung von mehrperiodigen Zahlungen, Einzige und einfach zu kommunizierende Kennzahl
	\item \textbf{Contra}: 
	\begin{itemize}
		\item Umständliche Berechnung
		\item \textbf{Problem bei einzelnem Projekt}: mehrere Lösungen, wenn mehr als ein Vorzeichenwechsel bei den Zahlungen stattfindet 
		$\rightarrow$ Gebe eine Zinssatzspanne vor, indem der Kapitalwert positiv ist, ansonsten nicht sinnvoll anzuwenden
		\item \textbf{Sich gegenseitig ausschließende Projekte}: 
		\begin{itemize}
			\item Unterschiedliche Größenordnungen kann dazu führen, dass die IZS-Methode inkonsistent zur KW-Methode ist (s. FS3/18)
			$\rightarrow$ Ziehe Projekt mit kleinerem $|C_0|$ vom anderen ab und entscheide auf Basis des Differenzenprojekts
			\item Kapitalwertmethode kann abhängig vom geforderten Zinssatz verschiedene Ergebnisse liefern (s. FS3/20), obwohl IZS gleich bleibt
			$\rightarrow$ Berechne neues Projekt B-A so, dass die Anfangsauszahlung negativ ist und berechne IZS; Wähle $B$, wenn $\text{geforderter Zinssatz} < IZS$, sonst $A$
		\end{itemize}
	\end{itemize}
\end{itemize}
\bigskip
\textbf{Methoden in der Realität}:
Kapitalwertmethode und IZSM liefern im Normalfall die richtige Entscheidungsvorlage. Wahl zusätzlich abhängig von der Unternehmensgröße und Branche:
\begin{itemize}
	\item Unternehmen mit vielen unterschiedlichen Segmenten: Eingeschränkte Vergleichbarkeit der Projekte und IZS aufgrund unterschiedlicher Risiken $\rightarrow$ Kapitalwertmethode
	\item Kapitalbeschränkte kleine Unternehmen: Projekte konkurrieren um begrenzte Ressourcen $\rightarrow$ Amortisationsrechnung
\end{itemize}
