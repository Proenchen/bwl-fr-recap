\section{Grundlagen des externen Rechnungswesens}

\textbf{Abgrenzung zur Finanzwirtschaft}:
\begin{itemize}
	\item Finanzwirtschaft: Fokus auf dem Wert von Investitionen über deren \underline{gesamte Lebensdauer}
	\item Rechnungswesen: Fokus auf wirtschaftlichem Erfolg eines Unternehmens während einer \underline{abgegrenzten Periode}
	\item Daten des Rechnungswesens im Gegensatz zu Zahlungsströmen nicht eindeutig!
\end{itemize}

\textbf{Reinvermögen und wirtschaftlicher Erfolg}:
\begin{itemize}
	\item \textbf{Reinvermögen} zum Zeitpunkt $t_i$ 
	\\= Wert der Vermögensgegenstände bei $t_i - \text{ Wert der Schulden bei } t_i$ 
	\item \textbf{Wirtschaftlicher Erfolg} der Periode $t_i$: Änderung des Reinvermögens zwischen Anfang ($t_{i-1}$) und Ende ($t_i$) der Periode 
\end{itemize}

\textbf{Erfolgsmessung auf Basis von Marktwerten}:
Wie können wir den Erfolg eines Unternehmens messen?
Ansätze:
\begin{itemize}
	\item Veränderung der Marktwerte der Vermögensgegenstände und Schulden\\
	Probleme bei Marktwerten von Vermögensgegenständen:
	\begin{itemize}
		\item Für viele VG irrelevant, da Unternehmen nicht vor hat, diese zu verkaufen
		\item Eindeutige Bestimmung oft nicht möglich
		\item Gefahr von subjektiven Schätzungen
		\item Unklar, wann Gewinn oder Verlust gebucht werden kann
	\end{itemize}
	Probleme bei Marktwerten von Schulden:
	\begin{itemize}
		\item Nur beobachtbar, wenn Schulden in Form von handelbaren Wertpapieren vorliegen
		\item Sinnlos, Schulden zum Marktwert zurückzukaufen (Zinseffekte)
	\end{itemize}
	\item Veränderung des Börsenwertes\\
	Probleme:
	\begin{itemize}
		\item Nur möglich, wenn Eigenkapital in Form von Aktien an Börsen gehandelt wird
		\item Henne-Ei-Problem: Was ist zuerst: Erfolg oder neuer Börsenwert?	
	\end{itemize}
	$\rightarrow$ Erfolgsmessung auf der Basis von Prinzipien des Rechnungswesens
\end{itemize}